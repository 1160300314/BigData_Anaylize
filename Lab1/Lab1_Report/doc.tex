\documentclass{ML}
% \setminted{fontsize=\scriptsize,baselinestretch=1}
\newminted{text}{frame=lines,framesep=2mm,fontsize=\scriptsize}
\newminted{python}{frame=lines, framesep=2mm, linenos}
% 姓名,学号
\infoauthor{朱明彦}{1160300314}

% 课程类型,实验名称
\infoexp{课程类型}{实验一}

\infoschool{计算机学院}{杨东华、王金宝}

\begin{document}
\maketitle

% \tableofcontents
\newpage

\begin{center}
    \textbf{\zihao{3} 实验一 \  数据预处理}
\end{center}

\section{实验目的}
掌握数据预处理的步骤和方法,包括数据抽样、数据过滤、数据标准化和归一化、数据清洗。理解数据预处理在各个步骤在大数据环境下的实现方式。

\section{实验环境}
\begin{itemize}
    \item Ubuntu 16.04.5
    \item Hadoop 2.7.7 伪分布配置
    \item Spark 2.4.0
\end{itemize}
\section{实验过程及结果}
% 对于实验内容3.1-3.4,请结合您的程序描述实验中各个数据预处理步骤的详细操作,
% 并说明每一步的输入和输出,包括输入输出数据的内容、格式等信息。
\subsection{数据抽样}
\paragraph{输入}数据样式如下,共$4783614$条数据,分层抽样比例$1\%$
% frame=lines 分割线
% linenos 行号
\begin{textcode}
144552912|9.349849|56.740876|17.052772|2011/06/27|18.5℃|83.91|38267|1974-06-08|Switzerland|programmer|5042
144552912|9.350188|56.740679|17.614840|2016-10-08|37.8℉|78.80|1205|1991-04-14|Italy|teacher|1705
144552912|9.350549|56.740544|18.083536|2010/05/19|14.0℃|80.73|28343|May 25,1989|Luxembourg|farmer|3208
144552912|9.350806|56.740485|18.279465|2014/10/19|2.6℃|80.52|36251|August 25,1992|Belgium|programmer|2565
144552912|9.351053|56.740486|18.422974|2017/12/10|1.2℃|77.90|27133|1992/03/25|Belgium|doctor|2455
144552912|9.351475|56.740502|19.124889|2018-12-01|26.2℉|82.83|25448|July 21,1997|France|Manager|2943
144552912|9.352127|56.740558|19.590593|April 18,2010|38.2℉|82.93|34087|1991/01/23|Germany|Manager|2984
144552912|9.352420|56.740597|19.621764|2016-01-09|-0.8℃|83.51|18577|April 21,1980|Denmark|programmer|3951
144552912|9.352584|56.740629|19.659931|2014-12-15|29.2℉|79.92|25939|1972/11/28|Holland|farmer|2208
144552912|9.352726|56.740663|19.490670|2013/11/05|-8.1℃|77.75|15206|1987-12-22|Italy|accountant|1788
\end{textcode}
\paragraph{输出}数据样式未变,如上面所示;抽样后的结果共有$47737$条数据。
\paragraph{抽样方法} 首先统计用于分层变量的\texttt{user\_carrer}所有可能的取值,并将所有的职业对应的分层比均置为$1\%$,利用分层抽样的思想即可抽得上述样本,具体如下代码所示。
\begin{pythoncode}
def parse(x):
    return x.split("|")
    
origin_record = sc.textFile("hdfs://.../input/large_data.txt").map(parse)
sample_record = origin_record.map(lambda x: (x[10], x))\
    .sampleByKey(False, dic, seed=seed).map(lambda x: x[1])
sample_record.map(lambda x: "|".join(x)).coalesce(1).saveAsTextFile("D_Sample")
\end{pythoncode}
\subsection{数据过滤}
\paragraph{输入} 有两个部分,其一为抽样结果\texttt{D\_Sample},共$47737$条数据;其二为原始数据即\texttt{D},共$4783614$条数据。
\paragraph{输出} 过滤筛选掉\texttt{longitude}不在$[8.1461259, 11.1993265]$,\texttt{latitude}不在$[56.5824856, 57.750511]$以及\texttt{rating}在前$1\%$和后$1\%$的无效数据,得到的结果共$4689245$条数据,
即\texttt{D\_Filtered}。
\paragraph{过滤方法} 将抽样数据\texttt{D\_Sample}按照\texttt{rating}进行排序,并提取前$1\%$以及后$1\%$对应的临界值,将提取后的临界值与经纬度坐标的范围同时作为过滤条件,对原始数据\texttt{D}进行过滤即可,具体见代码。
\begin{pythoncode}
def filter_func(x):
    lng = float(x[1])
    lat = float(x[2])
    ans1 = longitude[0] <= lng <= longitude[1]
    ans2 = latitude[0] <= lat <= latitude[1]
    ans3 = x[6] == '?' or (limit[0] <= float(x[6]) <= limit[1])
    return ans1 and ans2 and ans3

filtered_record_temp = sample_record.sortBy(lambda x: x[6]).collect()
length = len(filtered_record_temp)
outliers = length // 100
limit = float(filtered_record_temp[outliers][6]), 
            float(filtered_record_temp[-outliers][6])
filtered_record = origin_record.filter(filter_func)
filtered_record.map(lambda x: "|".join(x)).coalesce(1).saveAsTextFile("D_filtered")
\end{pythoncode}
\subsection{数据标准化和归一化}
\paragraph{输入} 经过过滤后的数据\texttt{D\_Filtered},数据格式如下
\begin{textcode}
144552912|9.349849|56.740876|17.052772|2011/06/27|18.5℃|83.91|38267|1974-06-08|Switzerland|programmer|5042
144552912|9.350188|56.740679|17.614840|2016-10-08|37.8℉|78.80|1205|1991-04-14|Italy|teacher|1705
144552912|9.350549|56.740544|18.083536|2010/05/19|14.0℃|80.73|28343|May 25,1989|Luxembourg|farmer|3208
144552912|9.350806|56.740485|18.279465|2014/10/19|2.6℃|80.52|36251|August 25,1992|Belgium|programmer|2565
144552912|9.351053|56.740486|18.422974|2017/12/10|1.2℃|77.90|27133|1992/03/25|Belgium|doctor|2455
144552912|9.351475|56.740502|19.124889|2018-12-01|26.2℉|82.83|25448|July 21,1997|France|Manager|2943
144552912|9.352127|56.740558|19.590593|April 18,2010|38.2℉|82.93|34087|1991/01/23|Germany|Manager|2984
144552912|9.352420|56.740597|19.621764|2016-01-09|-0.8℃|83.51|18577|April 21,1980|Denmark|programmer|3951
144552912|9.352584|56.740629|19.659931|2014-12-15|29.2℉|79.92|25939|1972/11/28|Holland|farmer|2208
144552912|9.352726|56.740663|19.490670|2013/11/05|-8.1℃|77.75|15206|1987-12-22|Italy|accountant|1788
\end{textcode}
\paragraph{输出} 将数据中温度的单位统一为\texttt{℃},并且统一所有日期的格式为\texttt{YYYY-MM-DD},将\texttt{rating}数据归一化,最终结果数据格式如下
\begin{textcode}
144552912|9.349849|56.740876|17.052772|2011-06-27|18.5℃|0.67|38267|1974-06-08|Switzerland|programmer|5042
144552912|9.350188|56.740679|17.614840|2016-10-08|20.0℃|0.53|1205|1991-04-14|Italy|teacher|1705
144552912|9.350549|56.740544|18.083536|2010-05-19|14.0℃|0.58|28343|1989-05-25|Luxembourg|farmer|3208
144552912|9.350806|56.740485|18.279465|2014-10-19|2.6℃|0.57|36251|1992-08-25|Belgium|programmer|2565
144552912|9.351053|56.740486|18.422974|2017-12-10|1.2℃|0.50|27133|1992-03-25|Belgium|doctor|2455
144552912|9.351475|56.740502|19.124889|2018-12-01|8.4℃|0.64|25448|1997-07-21|France|Manager|2943
144552912|9.352127|56.740558|19.590593|2010-04-18|20.4℃|0.64|34087|1991-01-23|Germany|Manager|2984
144552912|9.352420|56.740597|19.621764|2016-01-09|-0.8℃|0.66|18577|1980-04-21|Denmark|programmer|3951
144552912|9.352584|56.740629|19.659931|2014-12-15|11.4℃|0.56|25939|1972-11-28|Holland|farmer|2208
144552912|9.352726|56.740663|19.490670|2013-11-05|-8.1℃|0.50|15206|1987-12-22|Italy|accountant|1788
\end{textcode}
\paragraph{数据标准化和归一化方法} 简单利用规则进行\texttt{Map}即可,注意华氏度和摄氏度的转化方法以及日期的不同格式,具体见代码如下,其中第29行判断的是华氏度符号是否存在,此处显示有误。
\begin{pythoncode}
def time_temp_parse(x):
    standard_time_format = "%Y-%m-%d"
    unstandard_format = "%Y/%m/%d"
    unstandard_format_2 = "%B %d,%Y"
    review_date = x[4]

    temperature = x[5]
    review_date_time = None
    if '/' in review_date:
        review_date_time = datetime.strptime(review_date, unstandard_format)
        review_date = review_date_time.strftime(standard_time_format)
    elif ',' in review_date:
        review_date_time = datetime.strptime(review_date, unstandard_format_2)
        review_date = review_date_time.strftime(standard_time_format)
    x[4] = review_date

    user_birthday = x[8]
    user_birthday_time = None
    if '/' in user_birthday:
        user_birthday_time = datetime.strptime(
            user_birthday, unstandard_format)
        user_birthday = user_birthday_time.strftime(standard_time_format)
    elif ',' in user_birthday:
        user_birthday_time = datetime.strptime(
            user_birthday, unstandard_format_2)
        user_birthday = user_birthday_time.strftime(standard_time_format)
    x[8] = user_birthday

    if '℉' in temperature:
        temperature = "%.1f" % (float(temperature[:-1]) - 32 / 1.8) + "℃"
        x[5] = temperature

    rating = "%.2f" % ((float(x[6]) - limit[0]) / (limit[1] - limit[0]))\
                         if x[6] != "?" else x[6]
    x[6] = rating
    return x
normalization_record = filtered_record.map(time_temp_parse)
\end{pythoncode}
\subsection{数据清洗}
\paragraph{输入} 经过数据标准化和归一化之后的结果,经过统计共缺少$9444$个不同的信息,查询的方式如下面的\texttt{Shell}命令。
\begin{minted}{Shell}
    cat D_Filtered/part-* | grep "?" | wc -l 
\end{minted}
\paragraph{输出} 将所有的缺失信息进行填充之后的结果,即\texttt{D\_Preprocessed}。
\paragraph{数据填充方法} 主要分为以下两步,分别用于填充\texttt{user\_income}和\texttt{rating}。
\begin{enumerate}
    \item 根据\texttt{user\_career}和\texttt{user\_nationlity}进行分组,将每个分组的薪资\texttt{user\_income}平均值作为具有同样国籍和职业但缺失薪资的元组的填充值。
    \item 将根据第一步填充好的结果,利用机器学习中线性回归的方式和未缺失\texttt{rating}的数据中\texttt{longitude, latitude, altitude, user\_income}学习到相关的模型,对相应缺失值进行预测。
\end{enumerate}
\section{实验心得}
可以分享您在实验环境搭建、程序编写和调试以及结果分析过程中遇到的问题和解决方法。
% \appendix

% \section{源代码}
% \section{参考文献}
% \begin{thebibliography}{20}
    % \bibitem{employee_name} 中国最常见名字前50名, \texttt{https://www.sohu.com/a/164406113\_367620}
    % \bibitem{employee_id} 身份证号在线生成器, \texttt{https://www.tinysoft.org/}
% \end{thebibliography}

\end{document}
